\section{Architecture}
\label{section:architecture}

\begin{figure*}[t]
	\label{fig:architecture}
	\includesvg[width=\textwidth]{architecture.svg}
	\centering
	\caption[width=\textwidth]{The architecture of \textit{GESTALT} consists of the data collection subsystem, the ownership assignment process, the concept mapping process and the search subsystem.}
\end{figure*}

The architecture of \textit{GESTALT} is in Figure \ref{fig:architecture}. It decomposes into four essential functions: data acquisition, ownership assignment, concept mapping and search. 

\subsection{Data Acquisition}
- summary of the component purpose
- describe inputs and outputs
- describe the transformation


\subsection{Ownership Assignment}
- summary of the component purpose
- describe inputs and outputs
- describe the transformation


\subsection{Concept Mapping}
- summary of the component purpose
- describe inputs and outputs
- describe the transformation


\subsection{Search}
- summary of the component purpose
- describe inputs and outputs
- describe the transformation

\subsection{Summary of Architecture}
The Architecture of \textit{GESTALT} is designed to be lightweight and modular. The core requirement is to improve the ability to find locations of interest based on partial information. 
The search subsystem needs to balance precision in reducing the number of candidate locations with maximizing the recall of possible candidate locations. The recall is prioritized. 
The search space should be pruned with set membership checks based on the intuition that there is no point in running an expensive geospatial query over a location that doesn't contain the objects in question. 
Bloom filter checks are cheap; the human eye doesn't see invisible lines on a map. Accordingly, the ownership assignment subsystem can be inexact, and objects should be 'shared' between locations where appropriate. 
That location sharing in membership assignments improves the recall of the system. As a corollary, the subsequent spatial search process maintains the system's precision using the concept mappings of objects to extract implicit information about the location. 
Underlying the search problem requires collecting and processing objects and locations autonomously, at scale. Collecting and processing objects and locations is the first stage explored in section \ref{section:implementation}, implementation. 