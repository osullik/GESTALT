%Humans spend a lot of time searching for things. 
%With the advent of tools like google maps and open street maps, people can search through geospatial data at a whim. 
%These tools focus on providing exact matches to queries or a list of candidate locations based on the user's query. 
%Frequently, searchers only have access to partial information. 
%Whether it has been a long time since visiting a location, they have a vague recommendation from a friend or are an investigator trying to identify a location to solve a crime- a common problem is how to find a location of interest based on partial information. 
%This project designs \textit{the \textbf{G}eospatially \textbf{E}nhanced \textbf{S}earch with \textbf{T}errain \textbf{A}ugmented \textbf{L}ocation \textbf{T}argeting (\textbf{GESTALT})}, and implements a proof-of-concept of the proposed architecture. 
%Based on a new best-case dataset developed for this project, \textit{The Swan Valley Wineries dataset}, demonstrates the functionality and utility of \textit{GESTALT} while identifying substantial opportunities for future work. 


Geographic information systems (GIS) provide users with a means to efficiently search over spatial data given certain key pieces of information, like the coordinates or exact name of a location of interest. However, current GIS capabilities do not enable users to easily search for locations about which they have imperfect or incomplete information. In these cases, GIS tools may help with narrowing down to the general region of interest, but a manual last-mile search must then be performed by the user to find the exact location of interest within that region, which typically involves the visual inspection of remote sensing images or street-view images to identify distinct landmarks or terrain features that match the partial information known about the location. This step of the search process is a bottleneck, as it encumbers the user with the burden of sifting through many possible candidate locations until the correct one is visually identified. Taking inspiration from the way humans recall and search for information, we present \textit{the \textbf{G}eospatially \textbf{E}nhanced \textbf{S}earch with \textbf{T}errain \textbf{A}ugmented \textbf{L}ocation \textbf{T}argeting (\textbf{GESTALT})}, an end-to-end pipeline for extracting geospatial data, transforming it into coherent object-location relations, storing those relations, and searching over them. We contribute a new gold standard Swan Valley Wineries dataset and a proof of concept architecture that handles querying for spatial configurations of objects with respect to each other, handling uncertainty in the information known about a location, and accounting for the fuzzy boundaries between locations.