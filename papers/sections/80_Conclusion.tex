\section{Conclusion}
\label{section:conclusion}

\subsection{Future Work}
Throughout this paper, we identify many avenues for future work. Sections \ref{section:datasets} and \ref{section:architecture} explain the requirement for large-scale pictorial to geospatial scene mapping to enable the large-scale identification of objects to fuel \emph{GESTALT}'s search. 
Section \ref{section:datasets} highlights the need to develop datasets in dense suburban and urban locations to enable robust testing of the ownership assignment process. 
Sections \ref{section:architecture} and \ref{section:implementation} identify the requirement to trial the DVBSCAN algorithm to improve the ownership assignment process and the need to test the concept mapping proposed using KD-Trees robustly.
Sections \ref{section:implementation} and \ref{section:related} emphasize the need for a user query interface. Work on pictoral querying offers an exciting direction that enables abstracted user querying and leverages the cognitive advantages of geospatially constructing their query. 
The search component of \textit{GESTALT} needs to be tested at scale. While section \ref{section:implementation} notes that the assumption of only a \textit{last-mile} search allows us to assume small datasets, the performance of the belling tool discussed in section \ref{section:related} shows how quickly performance degrades if not managed well.
Finally, and most importantly, though the psychology literature indicates that the \textit{GESTALT} approach should be helpful to a searcher, there is no work evaluating this theory and measuring the extent to which it is functional. A user study should be prioritized for a fully functional \textit{GESTALT} prototype before it expands to full scale.

\subsection{Conclusion}
The \textit{GESTALT} project aims to reduce the time a searcher spends on the \textit{last-mile} of searching for a location. 
It assumes that the \textit{last-mile} is in a constrained geographical region and allows users to search for objects they are likely to remember from candidate locations. 
\textit{GESTALT} is designed to collect geospatial information about locations and objects within geographic regions from open-source geospatial and pictorial data. 
It infers the associations between objects visible to searchers in the real world and the location that they belong to and stores them using bloom filters and KD-Trees for efficient representation and search.
\textit{GESTALT} implements concept mapping to allow a user to query the implicit geospatial relations between objects in candidate locations, leveraging the inherent ordering of the multidimensional KD-Tree data structure for efficient search. 
\textit{GESTALT} demonstrates at a trim level, on the Swan Valley wineries dataset, that the approach is feasible and identifies future work in scaling it.


FUTURE: online clustering: \cite{Montiel2021}

Time decay on confidence score to account for changes in environment
Fuzzy string matching
Account for camera bearings to adjust coordinates of objects detected in images to be more precise
Incorporate user feedback to update tags that might be wrong in the data
Using object attributes to allow mroe detailed queries andstronger pruning



The first challenge is classifying an image as 'indoors' (where no objects will be visible from RSI and the closest building will 'own' it) or 'outdoors', where objects can map to the RSI. Numerous approaches exist to the indoor/outdoor scene classification \cite{Tong2017}. 
Each object in an outdoor photo's distance from the camera geolocation will be estimated using a myriad of depth estimation techniques \cite{Ming2021,Liu2020}. 
Where multiple images cover the same area from different perspectives, the composite of these images will be used to estimate the positions of objects, as has been shown in prior work like IM2GPS from Carnegie Mellon University \cite{Hays2008} and numerous other efforts over internet-available images \cite{Snavely2011}. 