\section{Architecture}
\label{section:architecture}

\begin{figure*}[t]
    \includesvg[width=\textwidth]{architecture.svg}
    \centering
    \caption[width=\textwidth]{The architecture of \emph{GESTALT} consists of the data collection subsystem, the ownership assignment process, the concept mapping process and the search subsystem.}
    \label{fig:architecture}
\end{figure*}

The architecture of \textit{GESTALT} is outlined in Figure \ref{fig:architecture}. The components cover four essential functions: data acquisition, object ownership assignment, concept mapping, and search. 
We briefly describe each of them below, focusing on their core purposes, and leaving the implementation details and motivation vis-a-vis human-centric search for sections \ref{section:data} through \ref{section:search}. 

\subsection{Data Acquisition}
The purpose of the Data Acquisition subsystem is to ingest heterogenous sources of objects and locations into \emph{GESTALT}, aiming to maximize the recall of all possible objects. 
\emph{GESTALT} currently supports ingestion of hand-labelled objects from KML Files, crowd-sourced object labels from OSM and automatically generated object tags we extract from images geolocated within our region of interest by Flickr.
All of these data sources are fused into a common format and assigned porbabalistic scores reflecting the likelihood that the object detected is actually present in the real world. 
The Data Acquisition subsystem ends when all of the data sources have been stored as JSON files, ready for ingestion by the Ownership Assignment subsystem. 

\subsection{Ownership Assignment}
The purpose of the Ownership Assignment subsystem is to identify which \textit{location} each identified \textit{object} belongs to. 
The system accepts a JSON file of \textit{locations} and a series of JSON files of geolocated \textit{objects} both generated by the Data Acquisition subsystem. 
We use the DBSCAN \osullikomment{cite dbscan} to cluster the objects prune out objects unlikely to be associated with any paticular location by assigning them to a \textit{Null Cluster}. 
The resulting cluster centroids are instantiated into a KD-Tree with the location centroids and a nearest-neighbour search determines what location is closest to that cluster of objects, and assigns it as the parent \textit{location} for that \textit{object cluster}.
The Ownership Assignment subsystem returns a dataframe of objects, their predicted parent locations and their respective coordinates. 

\subsection{Concept Mapping}
The purpose of the Concept Mapping subsystem is to create the data structures that will enable for advanced pruning and  geospatial searching of objects and locations, including the pictorial specification of queries, leveraging the human tendency to draw scratch-maps to describe locations and directions. 
We create three data structures: an inverted index, location-object index and a object-object matrix. 
The inverted index has the obejcts as keys and locations as values. It supports exact and fuzzy set membership querying to prune the search space for downstream geospatial searching by only returning locations that contain the objects being searched for. If the object doesn't exist at a location, there is no point in searching for its relative location.
The Concept Mapping subsystem accepts as input the dataframe of objects, their predicted parent locations and the coordinates of both. 
Using the input dataframe, the subsystem creates two distinct data structures, intended to support different types of spatial queries. 
The first location-object concept map treats the coordinates of the \textit{location} as the division point on both the north-south and west-east axes. 
For each location, the data structure contains four lexicographically ordered lists, one for each quadrant NW, NE, SW and SE. 
Each list contains the objects belonging to that location which reside in that quadrant, relative to the location centroid. 
The second object-object concept map makes no assumptions about the position of the objects relative to the location, and rather is a representation of each object, relative to every other object. 
The object-object concept map is sparse M x M matrix, where M is the number of objects assigned to the location, and an object at position [i,j] is the i$^th$ object from the north and the j$^th$ object from the west. 
The concept mapping subsystemreturns these three data structures - the object inverted index, the location-object indexes and the object-object matrices. 

\subsection{Search}
The purpose of the search subsystem is to enable the user to identify locations of interest based on the collection and geospatial arrangement of objects known to them. 
It accepts as input a user query - either through a keyword search or a pictorial query specification and the three data structures created by the concept mapping subsystem. 
It is capable of exact and fuzzy searching. 
The search subsystem balances precision in reducing the number of candidate locations with maximizing the recall of possible candidate locations. 
The recall is prioritized based on the probability that they are the user's intended location. 



\subsection{Summary of Architecture}
The Architecture of \textit{GESTALT} is designed to be lightweight and modular. The core requirement is to automate the last-mile search problem and allow users to search for locations of interest based on partial information. 
Achieving this outcome necessitates collecting and processing objects and locations autonomously, at scale.
Given a large quantity of noisy object tags, \emph{GESTALT} performs fuzzy Object assignment, allowing for objects to be assigned to multiple nearby locations, and thereby improving the recall of the system.
The subsequent concept mapping and spatial search processes maintain the system's precision using the relative directional relationships between query objects to extract implicit information about the location the user seeks. 