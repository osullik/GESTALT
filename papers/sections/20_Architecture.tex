\section{Architecture}
\label{section:architecture}

\begin{figure*}[t]
    \includesvg[width=\textwidth]{architecture.svg}
    \centering
    \caption[width=\textwidth]{The architecture of \emph{GESTALT} consists of the data collection subsystem, the ownership assignment process, the concept mapping process and the search subsystem.}
    \label{fig:architecture}
\end{figure*}

The architecture of \textit{GESTALT} is outlined in Figure \ref{fig:architecture}. The components cover four essential functions: data acquisition, object ownership assignment, concept mapping, and search. We briefly describe each of them below, focusing on their core purposes, and leaving the detailed implementation details and motivation vis-a-vis human-centric search for sections \ref{section:data} through \ref{section:search}. 

\subsection{Data Acquisition}
- summary of the component purpose

- describe inputs and outputs

- describe the transformation


\subsection{Ownership Assignment}
- summary of the component purpose

- describe inputs and outputs

- describe the transformation


\subsection{Concept Mapping}
- summary of the component purpose

- describe inputs and outputs

- describe the transformation

The search space is then pruned based on user-provided object-object or object-location directional constraints.
This avoids running the more expensive geospatial query over locations that do not contain the query objects.


\subsection{Search}
- summary of the component purpose

- describe inputs and outputs

- describe the transformation

The search subsystem balances precision in reducing the number of candidate locations with maximizing the recall of possible candidate locations. The recall is prioritized.

\subsection{Summary of Architecture}
The Architecture of \textit{GESTALT} is designed to be lightweight and modular. The core requirement is to automate the last-mile search problem and allow users to search for locations of interest based on partial information. 
Achieving this outcome necessitates collecting and processing objects and locations autonomously, at scale.
Given a large quantity of noisy object tags, \emph{GESTALT} performs fuzzy Object assignment, allowing for objects to be assigned to multiple nearby locations, and thereby improving the recall of the system.
The subsequent concept mapping and spatial search processes maintain the system's precision using the relative directional relationships between query objects to extract implicit information about the location the user seeks. 