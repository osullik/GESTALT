\section{Problem Definition}
\label{section:problem}

Describe the problem here......

\subsection{Preliminaries}
\textbf{Define region, location, object.....}
\emph{Objects} are any physical entity. For example, a \textit{tree, building, lake, bridge, gate} or \textit{sign} could be objects. \textit{Objects} can also have attributes that provide amplifying information about the object, including things like \textit{color, material, size, species etc.}. 

\emph{Locations} are physical entities that \textit{do} something. They are the meaningful grouping of objects determined by ownership, proximity or utility. They have some purpose other than being an object. They could be a \textit{business, attraction, property etc.}. 

\textbf{Define last-mile search, object ownership task, concept mapping, progressive search.....}


\subsection{Datasets}
\emph{GESTALT} requires a list of all \textit{objects} with their coordinates and a list of all \textit{locations} with their coordinates. These objects and locations can be obtained in a variety of ways, which we briefly outline below. 

\subsubsection{Objects}
\paragraph{Open Street Maps (OSM)}
\textit{Open Street Maps (OSM)} is a knowledge collective that contains open-source Geodata~\cite{Haklay2008}, which can be easily extended via their online editing interface.
%NSCH tighten this up
The primary concern with OSM is the accuracy and validity of crowd-sourced data \cite{VargasMunoz2020}. However, a combination of human review and an application of machine learning for detecting anomalous behavior in OSM edits \cite{Mooney2017} goes a long way toward addressing these concerns. 

OSM records objects' coordinates as a mixture of point coordinates and bounding polygons. 
OSM has a defined and curated ontology that defines the labeling scheme, maximizing interoperability. 
The OSM objects are crowd-sourced and of varying granularity and completeness. Incompleteness is part of the initial scope of OSM, with the founder noting that it's typically only what people want to add that gets added \cite{Haklay2008}.
In general, the completeness of OSM is unassured, so scaling \textit{GESTALT} beyond the trivial requires an automated method for object detection and resolution. 
Because of the lack of completeness and to enable evaluation with ground-truth labels, the Author used Google Earth to annotate objects present at six wineries within the Swan Valley Region of Western Australia. The wineries are separated in space and a semi-rural environment. The tags consist of an object name, its latitude \& longitude, and any descriptive markings written as key:value pairs. The objects are stored by their ground truth parent location in a KML file. Additional dataset creation for the small New South Wales town of Buladelah aims to fill a 'suburban' setting and of the Darby St Restaurant Strip in Newcastle, New South Wales, for an 'urban' environment. 

\paragraph{Object Detection}
Explain here......


\subsubsection{Locations}
%NSCH tighten this up
Google Maps\footnote{\href{https://www.google.com/maps/about/}{Google Maps}} maintains \textit{locations} as coordinate points with associated metadata. The locations are generally current and complete. 
Google Maps does not support bounding polygons at the location level; it appears to extend to as granular as ZIP Codes or suburb boundaries and no further. 
OSM Supports locations, but is less complete than Google Maps (at least for Australian Wine Regions.)
In general, for \textit{GESTALT} to function optimally, the input locations should be the union of Google Maps and OSM. However, given the limitations of Google Maps API usage, a dataset was manually curated in OSM using publicly available information and the Author's world knowledge. 
Creating the Swan Valley Winery dataset for this project has the benefit of yielding 31 additional nodes and associated metadata for the OSM project. 



%\subsection{Data Sources}
%Before delving into what data is needed, quickly reviewing the available data sources will highlight opportunities. 
%\subsubsection{Google Maps}
%Google Maps\footnote{\href{https://www.google.com/maps/about/}{Google Maps}} is a suite of applications maintained by Google for users to \textit{"explore and navigate} [their] \textit{world"}. 
%While it is open for community contributions of locations, it is a commercial platform that charges for API access\footnote{\href{https://mapsplatform.google.com/pricing/}{Pricing for Google Maps API}}. 
%In addition to coordinating information, Google Maps contains Street View imagery, user reviews, and user-submitted pictures. 
%Google Maps does not record \textit{objects} unless they are monuments or landmarks of significance.

%\subsection{Automation}
%Automating the labeling process is essential to scaling \textit{GESTALT} beyond a trivial size. 
%Options for automation are explored in detail in sections \ref{section:architecture} and \ref{section:related}, but essentially rely on combining remote sensing imagery, ground-based imagery and image metadata to generate mappings of objects to coordinates and parent locations.
%The ability to autonomously determine object locations will set the conditions for the remaining elements of GESTALT to scale.