\section{Search}
\label{section:search}

The search function has been implemented using the Python \textit{Pandas}\footnote{https://pypi.org/project/pandas/}{Pandas PyPI Repo} library. 
This approach assumes a single data frame of objects and their determined locations because of the number of possible attributes an object can have and the relatively few that they possess, this is a sparse data structure. 
The sparseness does indicate the discriminatory power of remembering attributes. For example, a 'door' is not informative, but a 'blue door' on your favourite seaside restaurant is more likely to prune the search space. 
Because \textit{GESTALT} is designed only for the last-mile search and assumes a small starting region, it may remain feasible to use a simple data structure like a Pandas data frame containing all the objects for all the locations for the query region. 
More work with the aggregation functions is required to determine if it can support all the necessary aggregation queries comparing object collections. 

Semantic search has not been implemented. However, the Levenshtein string distance metric (with $threshold = 0.8$) checks for small spelling discrepancies in input words. The priority weights towards retrieving all possible objects, so we accept the increased risk of mistakenly including an object to move the recall closer to 100\%. The next component to be implemented is a nearest-neighbour retrieval mechanism using word embeddings. Prior work indicates that developing databases of embeddings is trivial\cite{Mueller2012}, but using existing datasets tools like word2vec, GloVe and fasttext can generate embeddings over large, publicly available corpora that can be recreated. 

As discussed in the subsection on Ownership Assignment implementation, bloom filters are a much more efficient operator for set membership testing. 
The KD-Tree is more suited for geospatial queries, so the Pandas Dataframe currently supports the gaps between the two in supporting aggregation queries. More work is required to integrate these data structures into a coherent search pipeline that maximises recall while actively pruning the search space at every step so that the searcher can find their locations of interest. Natural language querying is an active area of research yet to present a solution capable of effectively translating natural language queries and their SQL solutions. Given the relatively constrained domain of this problem set, it is a good candidate for implementation as a low priority for improvement. 


%SCALABILITY
%Automating the labeling process is essential to scaling \textit{GESTALT} beyond a trivial size. 
%Options for automation are explored in detail in sections \ref{section:architecture} and \ref{section:related}, but essentially rely on combining remote sensing imagery, ground-based imagery and image metadata to generate mappings of objects to coordinates and parent locations.
%The ability to autonomously determine object locations will set the conditions for the remaining elements of GESTALT to scale.