\section{Concept Mapping}
\label{section:concept}

The core of \emph{GESTALT}'s geospatial search capability resides in what we call the \emph{Concept Mapping} component.
Concept mapping is the process of extracting and explicitly encoding the implicit geographic relationships between objects. 
By encoding these spatial relationships in a manner that can be compared with visual queries issued by the user, we enable a form of spatial last-mile search that (to our knowledge) does not exist in any GIS tools.
Concept mapping and querying consists of two phases: the encoding phase (offline) and the search phase (online). We discuss the details of the encoding phase in this section and leave the search details to section \ref{section:search}.
We implement two forms of concept mapping: object-location concept mapping and object-object concept mapping.

\subsection{Object-Location relations}
Object-location relations encode whether an object is North, South, East or West of a location. This type of information supports simple queries (i.e. when a user knows that a location has a lake on its western side).
\nrscomment{.walk through the diagram explaining how the process relates to how humans might think about search.}

\begin{figure*}[h]
    \centering
    \begin{subfigure}[t]{.3\textwidth}
        \includegraphics[width=\textwidth]{CM-ExampleLocation.png}
        \caption{\small A candidate location X has named objects A-D with the spatial layout depicted above.} 
        \label{fig:CM-LO-Example}
    \end{subfigure}
    \hfill
    \begin{subfigure}[t]{.3\textwidth}
        \includegraphics[width=\textwidth]{CM-LO-Setup.png}
        \caption{\small The objects are binned into spatial quadrants based on their relative position to the location centroid, X.} 
        \label{fig:CM-LO-Setup}
    \end{subfigure}
    \hfill
        \begin{subfigure}[t]{.3\textwidth}
        \includegraphics[width=\textwidth]{CM-LO-Query1.png}
        \caption{\small Searching counts the number of query terms that correspond to a location's quadrants and returns the count, allowing multiple candidate solutions to be ranked by closeness of match.}
        \label{fig:CM-LO-Query}
    \hfill
    \end{subfigure}
    \caption{\textbf{Generate and Query an Object-Location Concept Map.}}\label{figure:ConceptMap-LO} 
\end{figure*}




% This all belongs in search
%\subsubsection{Query Input}
%- algorithm for parsing from grid, etc.
%\subsubsection{Solution}
%- recursive algorithm for finding any match
%- complexity analysis
%
%\subsubsection{Experimental Results?}
%- on ground truth queries? report number of locations pruned? timings?



\subsection{Object-object relations}
- description....................walk through the diagram explaining how the process relates to how humans might think about search. use an example....................

\nrscomment{tighten up the chunk below and move some to query specification subsection below}
The representation of object-object relations as a concept map is grounded in the assumption that the provided query input is a pictorial query. 
When humans describe locations to each other, they frequently draw a map. 
We discuss pictorial querying further in section \ref{section:search}, but intuitively we want to be able to compare a user's sketch map to the real world locations \emph{GESTALT} records. 
Rather than use a fully-connected graph, we encode both the locations, and later, the pictorial query inputs as matrices with zeros representing unallocated space and the names of objects. 
The matrices are $NxN$, where $N$ is the number of objects at a location. 
Using algorithm \ref{alg:geoToGrid}, we assign each object in a given location to a position $(i,j)$ in the Matrix where $i$ is its order of appearance from north to south and $j$ is the same object's order of appearance from west to east. 
The result is a matrix in which each row and column has only a single object. 
Where ties are experienced in the real world (due to a lack of GPS precision, or very close position) they are broken lexicographically in the object-object concept map instantiation.
The process of creating and querying a concept map is shown in Figure \ref{figure:ConceptMap}.



\begin{figure*}[h]
    \centering
    \begin{subfigure}[t]{.3\textwidth}
        \includegraphics[width=\textwidth]{CM-ExampleLocation.png}
        \caption{\small A candidate location X has named objects A-D with the spatial layout depicted above.}
        \label{fig:CM-Example}
    \end{subfigure}
    \hfill
    \begin{subfigure}[t]{.3\textwidth}
        \includegraphics[width=\textwidth]{CM-OO-Setup.png}
        \caption{\small During pre-processing, the objects associated with location X are ordered North to South (NS) and West to East (WE) and mapped into a matrix with corresponding indices.}
        \label{fig:CM-OO-Setup}
    \end{subfigure}
    \hfill
        \begin{subfigure}[t]{.3\textwidth}
        \includegraphics[width=\textwidth]{CM-OO-Query1.png}
        \caption{\small Searching recursively prunes for ANY match. Each recursion is a darker shade, with unpruned area in white. Objects highlighted in yellow are found to match the query configuration; candidate location X is a match for the query.}
        \label{fig:CM-OO-Query}
    \hfill
    \end{subfigure}
    \caption{\textbf{Generate and Query an Object-Object Concept Map.}}\label{figure:ConceptMap} 
\end{figure*}


\subsection{Encoding the Spatial Query}
\nrscomment{ - screenshot of UI goes here and description of how a user would put objects in and move them around.
....discuss how pictorial input is the natural way to specify a query of this type.........}





This multiple-ownership situation is one of the driving requirements for implementing concept mapping to extract additional discriminatory information between locations based on the geospatial layout of objects.

%Concept mapping has been partially implemented in Python leveraging the \textit{Scipy} library\footnote{\href{https://pypi.org/project/scipy/}{SciPy PyPI Repo}}. Two different approaches have been trialed. 
%The first is simple dynamic arithmetic on the coordinates stored in a Pandas data frame. If one set of coordinates is above, below, left or right of another, it is north, south, east or west, respectively. 
%While these calculations are in constant time for straightforward comparisons of known objects (e.g. "is the pond west of the bridge"), the time complexity rapidly increases as soon as aggregations are employed. 
%Queries of "Give me everything west of the duck pond" would execute in $O(N)$ time as each element has to be examined. Worst-case queries would run in $O(N\sup{2})$ time, where every object is checked for its position relative to every other object. 

%The second (better) approach (only partially implemented) instantiates the objects within a location into a KD-Tree. 
%Assuming that the object centroid is the root, we can quickly complete queries like "Give me everything west of the duck pond" by leveraging the structure of the subtrees to return the requested set. 
%Similarly, getting the relative positions of two objects searches for a common ancestor. It uses the path between the children and their ancestor node to infer their spatial relation to each other.

%A third approach, designed to leverage the \textit{Neo4J Python Library}\footnote{\href{https://pypi.org/project/neo4j/}{Neo4J PyPI Repo}} to connect to a \textit{Neo4J Graph Database}\footnote{\href{https://neo4j.com/}{Neo4J Website}} but not implemented frames concept mapping as a graph traversal problem. 
%In this formulation, each object is a node on a graph. Weighted, labeled edges exist between each node within a given proximity threshold to the node. 
%The edge labels describe the neighboring node's cardinal direction and the distance weights. 
%After constructing the object graph, queries for 'give me everything west of the duck pond' would freely explore nodes connected by west, north and south edges. 
%It can only traverse along an east edge so long as the total cost of traveling east would be less than the cumulative value of the 'west' travel up to this point. 

%%Overall, concept mapping aims to enable geographic search over objects by explicitly representing their geospatial relationships to each other. 
%The author implemented a very basic approach using coordinate arithmetic was quickly determined to be infeasible for the extensive data sets that \textit{GESTALT} anticipates processing. 
%KD-Trees for the objects in each location have been implemented, as have the KD-Trees for the locations themselves. 
%This conceptual KD-Tree of KD-Trees approach performs a natural aggregation function which, provided that regions are created consistently, will allow for relative spatial queries at different levels of granularity. 
%Empirical evaluation of the performance of the arithmetic, KD-Tree and Graph-based approaches is yet to be completed. 



%%%%%%%%%%%%%%%NSCH clean up, incorporate, and delete the text below this point
%The first is \textit{Static Cardinal Relations} which encodes whether an object is North, South, East or West of a location. Static Cardinal Relations support simple queries where the user knows that a location has a lake on its western side. 
%The second is \textit{Dynamic Cardinal Relations} which determines whether an object is North, South, East or West of another object within a location. These queries support cases where a searcher might remember standing at a lake northeast of a location and that there was a swingset to the immediate west of them but still in the northeast of the location overall. 


%Concept mapping needs to be unsupervised and support aggregation. Tracking every object's relative location to every other object quickly becomes intractable, so mechanisms to aggregate depending on the level of granularity need to be applied. 
%Accordingly, the underlying data structure must support aggregation and relative position querying. 