\section{Datasets}
\label{section:datasets}
\textbf{(1.5 pages (7.5 Remain))}

There are two key datasets required for the GESTALT project. A list of all \textit{objects} with their coordinates and a list of all \textit{locations} with their coordiantes. 
Here, an \textit{object} is any physical entity that exists in the world. For example, a \textit{tree, building, lake, bridge, gate} or \textit{sign} could be objects.
\textit{Objects} can also have attributes, that provide amplifying information about the object, including things like \textit{color, material, size, species ... etc}. 
Locations are physical entities that \textit{do} something. They have some purpose other than being an object. They could be a \textit{business, attraction, property... etc}. They are the meaningful grouping of objects, determined by owenership, proximity or utility. 

\subsection{Data Sources}
Before delving into what data is specifically needed, a quick review of the available data sources will highlight opportunities. 

\subsubsection{Google Maps}
Google Maps\footnote{\href{https://www.google.com/maps/about/}{Google Maps}} is a suite of applications maintained by Google for users to \textit{"explore and navigate} [their] \textit{world"}. 
While it is open for community contributions of locations, it is a commerical platform that charges for API access\footnote{\href{https://mapsplatform.google.com/pricing/}{Pricing for Google Maps API}}. 
In addition to coordinate information, Google Maps contains Street View imagery, user reviews and user submitted pictures. 

\subsubsection{Open Street Maps (OSM)}
The open-source alternative to Google Maps, \textit{Open Street Maps (OSM)} is a \textit{"knowledge collective that creates Open Geodata as its main objective"}\cite{Haklay2008}. 
The data already contained within OSM is open-source, and the only requirement for adding data is registering with a user account. Their online editing interface makes adding data easy. 
The major concern with OSM is the accuracy and validity of crowd-sourced data \cite{VargasMunoz2020}, however a combination of human review an application of machine learning for detecting anomalous behaviour in OSM edits \cite{Mooney2017} goes a long way toward addressing these concers.  

\subsection{Objects}
For \textit{GESTALT} to scale, it needs to have access to the coordinates of all \textit{objects} within an area of interest. 
Google Maps does not record \textit{objects} unless they are monuments or landmarks of significance. 
OSM does record the coordinates of objects, as a mixture of point coordinates and bounding polygons. 
OSM has a defined and curated ontology that defines the labeling scheme, maximizing interoperability. 
The OSM objects are crowd-sourced, and of varying granularity and completeness. Incompleteness is part of the initial scope of OSM, with the founder noting that its typically only what people want to add that gets added \cite{Haklay2008}.
In general, the completeness of OSM cannot be assured, so scaling \textit{GESTALT} beyond the trivial requires an automated method for object detection and georesolution. 
Because of the lack of completeness, and to enable evluation with ground-truth labels, the Author used Google Earth to annotate objects present at six wineries within the Swan Valley Region of Western Australia. The wineries are seperated in space and in a semi-rural environment. Additional dataset creation for the small New South Wales town of Buladelah aims to fill a 'suburban' setting and of the Darby St Restaurant Strip in Newcastle, New South Wales for an 'urban' setting. The tags consist of an object name, its latitude \& longitude, any descriptive markings written as key:value pairs. The objects are stored by their ground truth parent location in a KML file. 


\subsection{Locations}
For \textit{GESTALT} to scale, it needs to have access to the coordinates of all \textit{locations} within an area of interest. 
Google Maps maintains \textit{locations} as coordinate points with associated metadata. The locations are generally current and complete. 
Google Maps does not support bounding polygons at the location level, it appears to extend to as granular as ZIP Codes or suburb boundaries and no further. 
OSM Supports locations, but in my experimentation it is less complete than Google Maps (at least for Australian Wine Regions.)
In general, for \textit{GESTALT} to function optimally, the union of Google Maps and OSM should be used. However, given the limitations on Google Maps' API usage, a dataset was manually curated in OSM using publicly available information and the Author's world knowledge. 
The creation of the Swan Valley Winery dataset for this project has the additional benefit of yielding 31 additional nodes and associated metadata for the OSM project. 

\subsection{Automation}
The automating the labelling process is essential to scaling \textit{GESTALT} beyond a trivial size. 
Options for automation are explored in detail in sections \ref{section:related} and \ref{section:limitations}, but essentially rely on combining remote sensing imagery, ground based imagery and image metadata to generate mappings of objects to coordinates and parent locations.
The ability to autonomously determine object locations will set the conditions for the remaining elements of GESTALT to  



