\section{Introduction}
\label{section:introduction}
\textbf{(3/4 of a page (9 Remain))}

\subsection{Background}
When a user of a geospatial information system (GIS) is searching for a specific location, they have many tools at their disposal. 
If they remember the exact name of the location they are looking for,
most modern systems can return an exact match. 
If they know the physical address of the location, they can use that information to
determine what they are looking for.
There are situations, however, where a user has access to imper-
fect information about a location they are searching for. 
Perhapsit is a location they visited a long time ago which has faded in
memory or a vague recommendation from a friend. 
Perhaps they attempt to combine fragmented evidence for a police investigation
or fuse intelligence in a military operation.
Regardless of the reason, common approaches to the problem
now see users employing GIS tools for their bold adjust to get them
to the correct general area and then rely on a manual last mile effort
typically involving the visual inspection of remote sensing imagery,
searching for distinct landmarks or terrain features that match their
partial information. 
The last mile effort, while the most important, is also the bottleneck.
My proposed project asks is it possible to augment existing GIS
systems with micro-terrain data, to elevate the bottleneck of the ’last
mile’ search.

\subsection{Overview of Running-Example}
Throughout the \textit{GESTALT} paper, a running scenario illustrates the architecture in use. 
The running scenario is a \textit{last-mile search} problem. Their task is to find out which of the wineries they visited before attending a lunch with friends to share their recommendations. 
However, because of over-indulgence while on the tour, they are struggling to recall exactly which wineries they visited.
They know that the general region is the Swan Valley Wine Region near the city of Perth in Western Australia. There are more than 30 wineries in the small Swan Valley Wine Region, and far more locations when distilleries, breweries, chocolate shops and other auxiliary venues are included. 
They remember a few key facts about their trip, but the details are fuzzy. That is, they have \textit{partial information} about each of the wineries:
\begin{enumerate}
	\item They have a photo in front of Oakover Grounds, so know they must have visited it. 
	\item They recall drinking a lovely Verdehlo at a table made from wine barrells. 
	\item They didn't like the Chenin Blanc they were drinking in the place with palm trees.
	\item There was a noisy generator at one that ruined the ambience. 
\end{enumerate}	
They were able to find Oakover Estate easily enough using Google Maps, but didn't recognise any of the other names. 
When searching for "Winery with wine barrell tables" they recieved incolclusive results, as most of the text associated with the wineries mentions wine, barrells and tables. 
Ordinarily, they would manually use google maps satellite imagery, street view and user review photos to identify which wineries they were at. 
Surely there is a better way.

